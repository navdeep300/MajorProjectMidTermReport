
\section{Software Requirement Analysis}
A Software Requirements Analysis for a software system is a complete 
description of the behaviour of a system to be developed. It includes 
a set of use cases that describe all the interactions the users will 
have with the software. In addition to use cases, the SRS also contains 
non-functional requirements. Non-functional requirements are 
requirements which impose constraints on the design or implementation.
\begin{itemize}
\item{\bf General Description}: The Drawing module allows you to put your 3D work on paper. That is, to put views of your models in a 2D window and to insert that window in a drawing, for example a sheet with a border, a title and your logo and finally print that sheet. The drawing may consist of different views like top, front, side and orthographic views. It is developed using FreeCad, Qt and C++.
\item{\bf Users of the System}: The main target users are the Civil Engineers who want their plans to be printed on the sheets.
As of now, they have to create the drawings separately with different views in any CAD software
and the 3D model separately. So to automate converting a particular three-dimensional model to
the print-ready drawings (with different views), this project will be beneficial. So to decrease the
efforts, time and cost, it would be really beneficial.
\end{itemize}
\subsection{Functional Requiremets}
\begin{itemize}
\item {\bf Specific Requirements}: This phase covers the whole requirements 
for the system. After understanding the system we need the input data 
to the system then we watch the output and determine whether the output 
from the system is according to our requirements or not. So what we have 
to input and then what we’ll get as output is given in this phase. This 
phase also describe the software and non-function requirements of the 
system.
\item {\bf Input Requirements of the System}
\begin{enumerate} 
\item Three Dimensional Model.
\item Dimensions Of Different Views.
\item Scaling Parameters.
\item Name Of Particular Object.
\end{enumerate}
\vskip 0.5cm
\item {\bf Output Requirements of the System}
\begin{enumerate} 
\item 2Dimensional Model.
\item Generation Of Front View.
\item Generation Of Top View.
\item Generation Of Side View.
\end{enumerate}
\vskip 0.5cm
\item {\bf Special User Requirements}
\begin{enumerate} 
\item Exporting the 2D model in pdf or svg format.
\end{enumerate}
\vskip 0.5cm
\item {\bf Software Requirements}
\begin{enumerate} 
\item Programming language: C++
\item Framework: Qt
\item Documentation: Doxygen 1.8.3
\item Text Editor: Gedit, Notepad++, Sublime
\item Operating System: Ubuntu 12.04 or up
\item Web Server: Apache 2.4
\end{enumerate}
\end{itemize}
\subsection{Non functional requirements}
\begin{enumerate} 
\item Scalability: System should be able to handle a number of users. 
For e.g., handling around hundred users at the same time.
\item Usability: Simple user interfaces that a layman can understand.
\item Speed: Speed of the system should be responsive i.e. Response to
 a particular action should be available in short period of time. 
\end{enumerate}







