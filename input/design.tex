\section{Product Perspective}
Drawing automation refers to the varied computer machinery and software used to digitally create, collect, store, manipulate, and relay information needed for accomplishing basic drawing tasks. Creating two dimensional models, electronic transfer, and the management of automated models comprise the basic activities of an drawing automation system.Drawing automation helps in optimizing or automating existing drawing procedures. \\

\noindent Automated Building Drawing is a project for creating two-dimensional drawings (front-view, top-
view, side-view etc.) from a three-dimensional model. The main purpose or objective of the project is to make it usable even by the layman. The main
target users are the Civil Engineers who want their plans to be printed on the sheets. As of now,
they have to create the drawings separately with different views in any CAD software and the 3D
model separately. So to automate converting a particular three-dimensional model to the print-
ready drawings (with different views), this project will be beneficial. The interface should be easy
to use and pretty intuitive. Because the interface is a thing that makes user experience better and
to make the user use it.


\section{User Characteristics}
The objective of this system is to provide an efficient and effective service to the students and teachers or any other persron related to drawing directly or indirectly. It is aimed to encourage people to use computers and internet instead of paper and pen for their daily work. The Drawing module allows user to put your 3D work on paper. That is, to put views of your
models in a 2D window and to insert that window in a drawing, for example a sheet with a border,
a title and your logo and finally print that sheet. The drawing may consist of different views like
top, front, side and orthographic views.


\section{System Design} Systems design is the process or art of defining 
the architecture, components, modules, interfaces, and data for a 
system to satisfy specified requirements. One could see it as the 
application of systems theory to product development. There is some 
overlap with the disciplines of systems analysis, systems architecture 
and systems engineering.
\begin{itemize}
\item  External design: External design consists of conceiving, 
planning out and specifying the externally observable characteristics 
of the software product. These characteristics include user displays 
or user interface forms and the report formats, external data sources 
and the functional characteristics, performance requirements etc. 
External design begins during the analysis phase
and continues into the design phase.
\item  Logical design: The logical design of a system pertains to an 
abstract representation of the data flows, inputs and outputs of the 
system. This is often conducted via modeling, which involves a 
simplistic (and sometimes graphical) representation of an actual 
system. In the context of systems design, modeling can undertake the 
following forms, including:
\begin{itemize}
\item Data flow diagrams
\item Entity Relationship Diagrams
\end{itemize}
\item  Physical design: The physical design relates to the actual 
input and output processes of the system. This is laid down in terms 
of how data is input into a system, how it is verified/authenticated, 
how it is processed, and how it is displayed as output.
\end{itemize}
\section{Technologies Used}
\subsection{C++}
\noindent C++ is one of the most popular programming languages and is implemented on a wide variety of
hardware and operating system platforms. As an efficient compiler to native code, its application
domains include systems software, application software, device drivers, embedded software, high-
performance server and client applications, and entertainment software such as video games. Sev-
eral groups provide both free and proprietary C++ compiler software, including the GNU Project,
Microsoft, Intel and Embarcadero Technologies. C++ has greatly influenced many other popular
programming languages, most notably C\# and Java. Other successful languages such as Objective-
C use a very different syntax and approach to adding classes to C.\\

\noindent Bjarne Stroustrup began his work on ”C with Classes” in 1979. The idea of creating a new language
originated from Stroustrup’s experience in programming for his Ph.D. thesis. Stroustrup found that
Simula had features that were very helpful for large software development, but the language was
too slow for practical use, while BCPL was fast but too low-level to be suitable for large software
development. When Stroustrup started working in AT\&T Bell Labs, he had the problem of analyz-
ing the UNIX kernel with respect to distributed computing. Remembering his Ph.D. experience,
Stroustrup set out to enhance the C language with Simula-like features. C was chosen because it
was general-purpose, fast, portable and widely used. Besides C and Simula, some other languages
that inspired him were ALGOL 68, Ada, CLU and ML. At first, the class, derived class, strong
type checking, inlining, and default argument features were added to C via Stroustrup’s C++ to C
compiler, Cfront. The first commercial implementation of C++ was released on 14 October 1985.\\

\subsection{Introduction To Qt}
Qt Creator is a complete IDE for creating applications with Qt Quick and the Qt application framework.
Qt  is  designed  for  developing  applications  and  user  interfaces  once  and  deploying  them  across  several
desktop and mobile operating systems.
One  of  the  major  advantages  of  Qt  Creator  is  that  it  allows  a  team  of  developers  to  share  a  project
across  different  development  platforms  (Microsoft  Windows,  Mac  OS  X,  and  Linux)  with  a  common
tool  for  development and debugging. In addition, UI  designers can join the team by using Qt Quick tools
for creating fluid user interfaces in close cooperation with the developers.
The  main  goal  for  Qt  Creator  is  meeting  the  development  needs  of  Qt  Quick  developers  who  are
looking  for  simplicity, usability, productivity, extendibility and openness,  while aiming  to lower the barrier
of  entry  for  newcomers  to  Qt  Quick  and  Qt.  The  key  features  of  Qt  Creator  allow  UI  designers  and
developers to accomplish the following tasks:
\begin{itemize}
\item Get  started  with  Qt  Quick  application  development  quickly  and  easily  with  examples,  tutorials,
and project wizards.
\item Design  application  user  interface  with  the  integrated  editor,  Qt  Quick  Designer,  or  use graphics
software to design the user interface and use scripts to export the design to Qt Quick Designer.
\item Develop  applications   with  the   advanced  code  editor  that  provides  new  powerful  features  for
copleting code snippets, refactoring code, and viewing the element hierarchy of QML files.
\item Build  and  deploy  Qt  Quick  applications  that  target  multiple  desktop and mobile platforms, such
as Microsoft Windows, Mac OS X, Linux, Symbian, MeeGo, and Maemo.
\item Debug  JavaScript  functions  and  execute  JavaScript  expressions  in  the  current  context,   and
inspect QML at runtime to explore the object structure, debug animations, and inspect colors.
\item Profile  your  Qt  Quick  applications  with  the  QML  Profiler.  You can inspect binding evaluations,
signal  handling,  and  painting  operations  when  running  QML  code.  This  is  useful  for  identifying
potential bottlenecks, especially in the evaluation of bindings.
\item Deploy  applications  to  mobile  devices  and  create  application  installation  packages  for  Symbian
and Maemo devices that can be published in the Ovi Store and other channels.
\item Easily access information with the integrated context­sensitive Qt Help system.
\item It has differents modes such as Welcome, edit debug, design,analyze and help
\end{itemize}
\subsection{Introduction to Github}
\noindent GitHub is a Git repository web-based hosting service which offers all of the functionality of Git as well as adding many of its own features. Unlike Git which is strictly a command-line tool, Github provides a web-based graphical interface and desktop as well as mobile integration. It also provides access control and several collaboration features such as wikis, task management, and bug tracking and feature requests for every project.\\

\noindent GitHub offers both paid plans for private repto handle everything from small to very large projects with speed and efficiency. ositories, and free accounts, which are usually used to host open source software projects. As of 2014, Github reports having over 3.4 million users, making it the largest code host in the world.\\

\noindent GitHub has become such a staple amongst the open-source development community that many developers have begun considering it a replacement for a conventional resume and some employers require applications to provide a link to and have an active contributing GitHub account in order to qualify for a job.

\subsection{What is Git?}
\noindent Git is a distributed revision control and source code management (SCM) system with an emphasis on speed, data integrity, and support for distributed, non-linear workflows. Git was initially designed and developed by Linus Torvalds for Linux kernel development in 2005, and has since become the most widely adopted version control system for software development.\\

\noindent As with most other distributed revision control systems, and unlike most client–server systems, every Git working directory is a full-fledged repository with complete history and full version-tracking capabilities, independent of network access or a central server. Like the Linux kernel, Git is free and open source software distributed under the terms of the GNU General Public License version 2 to handle everything from small to very large projects with speed and efficiency.\\

\noindent Git is easy to learn and has a tiny footprint with lightning fast performance. It outclasses SCM tools like Subversion, CVS, Perforce, and ClearCase with features like cheap local branching, convenient staging areas, and multiple workflows.

\subsubsection{Installation of Git}

Installation of git is a very easy process.
The current git version is: 2.0.4.
Type the commands in the terminal:\\\\
\emph{
\$ sudo apt-get update\\\\
\$ sudo apt-get install git\\\\}
This will install the git on your pc or laptop.

\subsubsection{Various Git Commands}

Git is the open source distributed version control system that facilitates GitHub activities on your laptop or desktop. The commonly used Git command line instructions are:-\\

\subsubsection{Create Repositories}
%\addcontentsline{toc}{subsection}{Create Repositories}
Start a new repository or obtain from an exiting URL

\begin{description}

\item [\$ git init [ project-name]]\\
Creates a new local repository with the specified name
\item [\$ git clone [url]]\\
Downloads a project and its entire version history

\end{description}


\subsubsection{Make Changes}
%\addcontentsline{toc}{subsection}{Make Changes}
Review edits and craft a commit transaction

\begin{description}

\item [\$ git status] \leavevmode \\
Lists all new or modified files to be committed

\item [\$ git diff] \leavevmode \\
Shows file differences not yet staged

\item [\$ git add [file]]\\
Snapshots the file in preparation for versioning

\item [\$ git reset [file]]\\
Unstages the file, but preserve its contents

\item [\$ git commit -m "[descriptive message]"]\\
Records file snapshots permanently in version history

\end{description}
\leavevmode \\


